\documentclass[dvipdfmx,12pt]{beamer}% dvipdfmxしたい
%
\usepackage{amsmath,amssymb}
\usepackage{bm}
\usepackage{graphicx}
\usepackage{ascmac}
% pLaTeX文書; 文字コードはいつも通り
\usepackage{bxdpx-beamer}% dvipdfmxなので必要
\renewcommand{\kanjifamilydefault}{\gtdefault}% 既定をゴシック体に
\usefonttheme{professionalfonts}
% あとは欧文の場合と同じ
\usetheme{Copenhagen}

%
\title{ABC169-D : Div Game 解説}
\author{ホスフィン \\ twitter: @mine691}
\date{\today}
\begin{document}
\maketitle
%
%
\section{D - Div Game}
\begin{frame}
\begin{block}{問題文}
正の整数 $ N $ が与えられます. $ N $ に対して, 以下の操作を繰り返し行うことを考えます.
	\begin{itemize}
		\item はじめに. 以下の条件を全て満たす正の整数 $ z $ を選ぶ
		\begin{itemize}
				\item ある素数 $ p $ と正の整数 $ e $ を用いて, $ z = p ^ e $ と表せる.
				\item $ N $ が $ z $ で割り切れる.
				\item 以前の操作で選んだどの整数とも異なる.
		\end{itemize}
		\item $ N $ を $ N / z $ に置き換える.
	\end{itemize}
最大で何回操作を行うことができるか求めてください.
\end{block}
\begin{exampleblock}{制約}
	\begin{itemize}
		\item 入力はすべて整数
		\item $1 \leq N \leq 10 ^ {12}$
	\end{itemize}
\end{exampleblock}
\end{frame}

\begin{frame}
\begin{exampleblock}{ $N$ が素数冪のとき}
	$ N = p ^ e $ なら, $ z =  p ^ 1, p ^ 2, p ^ 3 \cdots $ としていくのが最適 (冪の小さい順に貪欲)\\
	$ \because ) $ 選ぶ冪を小さくすることで,操作回数が増やせることがある
\end{exampleblock}

$ N $ が素数冪でなくても, 素数毎に独立に考えることができる.\\
\end{frame}

\begin{frame}
$ N $ を素因数分解してシミュレーション \\
時間計算量は $ O ( \sqrt{N} ) $ \\

\href{https://atcoder.jp/contests/abc169/submissions/13986174}{実装例}
\end{frame}

\begin{frame}{統計情報}
人数 : 4591 / 6192 \\
正解率 : 74.14% \\
平均ペナ : 1.13 \\
ペナ率 : 46.50% \\ 
difficulty : 686 \\ 

\end{frame}
%
%
\end{document}
