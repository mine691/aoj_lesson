\documentclass[dvipdfmx,12pt]{beamer}% dvipdfmxしたい
%
\usepackage{amsmath,amssymb}
\usepackage{bm}
\usepackage{graphicx}
\usepackage{ascmac}
% pLaTeX文書; 文字コードはいつも通り
\usepackage{bxdpx-beamer}% dvipdfmxなので必要
\usepackage{hyperref} %リンクを有効にする

\renewcommand{\kanjifamilydefault}{\gtdefault}% 既定をゴシック体に
\usefonttheme{professionalfonts}
% あとは欧文の場合と同じ
\usetheme{Copenhagen}

%
\title{ABC170-F 解説}
\author{ホスフィン @mine691}
\date{\today}
\begin{document}
\maketitle
%

\section{F問題}
\begin{frame} % スライド1pごとに frame を作る
\begin{block}{問題文}
% ココに問題文を書く
$ W \times H $ のグリッドが与えれられる.(移動不可能なマスもある)
$ (x_1, y_1) $ から出発して, $ (x_2, y_2) $ に移動することを考える.
このとき,コスト1で上下左右1方向に最大 $ K $マスまで移動できる.
移動コストの最小値を求めよ.
\end{block}


\begin{exampleblock}{制約}
% ココに制約を箇条書きで書く

	\begin{itemize}
		\item 入力はすべて整数
		\item $ 1 \leq H, W, K \leq 10 ^ {6} $
		\item $1 \leq H \times W \leq 10 ^ {6} $
	\end{itemize}

\end{exampleblock}

\end{frame}

\begin{frame}{簡単な考察}
愚直にやってみると,$ O(HWK) $ かかりそうでダメ \\
「進行方向を変えたとき」と「K以上進んだとき」でコストが増加する\\
dist[x][y][direction] の形で(最短経路, 進んだ回数)を求めると良さそう (ダイクストラしたい気持ちが抑えきれない)\\
\end{frame}

\begin{frame} {解法例}
\begin{itemize}
	\item 拡張ダイクストラ  $ O(HW \log (HW)) $
	\item 01-BFS  $ O(HW) $
	\item 枝刈り(?) BFS  $ O(HW) $
	\item noshi 式区間に辺を貼るテク  $ O(HW \log K) $
	\item set で頑張るやつ  
\end{itemize}
今回は上3つを説明する.
\end{frame}

\begin{frame}{拡張ダイクストラとは?}
正確には,「拡張(されたグラフ上で)ダイクストラ」のこと\\
ダイクストラをするとき,最短経路と一緒に別の情報を持っておくことで,色々な量を一緒に求めることができる \\
priority\_queue を使うためには,「優先度」がついたものでないといけない \\
本問題では,\\
\begin{center}
(最短経路長, ある方向に何回連続で進んだか)
\end{center}
を優先度にする.
\end{frame}

\begin{frame}{拡張ダイクストラ解法}
説明がめんどくさいから \href{https://atcoder.jp/contests/abc170/submissions/14452259}{コード} を見てくれ
\end{frame}

\begin{frame}{01-BFS解法}
グリッド上で,辺のコストが 0 or 1 の場合,01-BFS というアルゴリズムが使える \\
deque (両端キュー) を使うと $ O(HW) $ \\
01-BFSについては「01-BFSのちょっと丁寧な解説」で検索!\\

\href{https://atcoder.jp/contests/abc170/submissions/14447856}{コード}
\end{frame}

\begin{frame}{枝刈りBFS解法}
実は,愚直でも工夫すれば通る\\
$K$ 進むとき,\\
\begin{itemize}
	\item 自分より大きいコストがあれば無視して進む
	\item 自分以下のコストが現れたら終わり
\end{itemize}
をすれば,各マスは高々2回 or 4回しか見ない

全体として $ O(HW) $ でできる\\
\href{https://atcoder.jp/contests/abc170/submissions/14447887}{コード}
\end{frame}

\begin{frame}{その他の解法}
noshi 式区間に辺を貼るテク\\
set で頑張るやつ : 未到達のマスを set で持つ.実装で破滅しがち.
\end{frame}
%
%
\end{document}