\documentclass[dvipdfmx,12pt]{beamer}% dvipdfmxしたい
%
\usepackage{amsmath,amssymb}
\usepackage{bm}
\usepackage{graphicx}
\usepackage{ascmac}
% pLaTeX文書; 文字コードはいつも通り
\usepackage{bxdpx-beamer}% dvipdfmxなので必要
\usepackage{hyperref} %リンクを有効にする

\renewcommand{\kanjifamilydefault}{\gtdefault}% 既定をゴシック体に
\usefonttheme{professionalfonts}
% あとは欧文の場合と同じ
\usetheme{Copenhagen}

%
\title{ABC171-F Strivore 解説}
\author{@mine691}
\date{\today}
\begin{document}
\maketitle
%

\section{問題概要}
\begin{frame} % スライド1pごとに frame を作る
\begin{block}{問題文}
% ココに問題文を書く
「好きな英小文字 $1$ 文字を好きな位置に挿入する」という操作を文字列 $S$ にちょうど $K$ 回繰り返してできる文字列は何通りあるでしょう?
\end{block}

\begin{exampleblock}{制約}
% ココに制約を箇条書きで書く

	\begin{itemize}
		\item 入力はすべて整数
		\item $1 \leq K \leq 10 ^ {6}$
		\item $1 \leq | S |  \leq 10 ^ {6}$
	\end{itemize}

\end{exampleblock}
\end{frame}

\section{考察}
\begin{frame}
\begin{exampleblock}{具体例}
	$ S = $  ab, $ K = 1 $  のとき \\
	\begin{itemize}
	\item *ab
	\item a*b
	\item ab*
	\end{itemize}
	のような形があって $ 76 $ 通り
\end{exampleblock}

\begin{alertblock}{注意}
上の例で aab にする方法は $ 2 $ 通りある
\begin{itemize}
	\item 1文字目の前に a を挿入する
	\item 2文字目の前に a を挿入する
	\end{itemize}
\end{alertblock}
操作の手順や方法を考えずに解きたい!
\end{frame}

\begin{frame}
\begin{alertblock}{問題文の言い換え}
操作が終了した文字列を $ T $ としたとき \\
\begin{center}
「文字列 $ S $ を部分列として含む文字列 $ T $ は何通りあるか」\\
ただし,$ |T| = |S| + K $ である
\end{center}
という問題に言い換えられる
\end{alertblock}
$S =$ ab, $K = $ 1 のとき \\
aab は ab を部分列に持つ \\
以下では,$ N = |S|$ とする

\end{frame}

\begin{frame}
ところで, 文字列$ T $ が文字列 $ S $ を部分列に持つことはどうやって判定できるだろうか\\
$ \longrightarrow $ 先頭から貪欲に一致判定していく \\

\begin{exampleblock}{貪欲の例}
	$ T = $ aabcababcdf, $ S = $ abcd のとき,先頭から S  の文字があるかを見る 
\end{exampleblock}

この構造を見ると \\
\begin{center}
「文字列$ T $ が文字列 $ S $ を部分列に持つ」\\
$ \Leftrightarrow $ \\
「$ T = t_1 s_1 t_2 s_2 \cdots s_N t_{N+1} $ となる文字列 $ t_i (1 \leq i \leq N + 1) $ が存在して,文字列 $ t_i (1 \leq i \leq N)$ は文字 $ s_i $ を含まない」
\end{center}
である
\end{frame}

\begin{frame}
\begin{block}{解くべき問題}
$ S $ から $ K $ 回操作をして,$ T $ になったとする \\
「$ T = t_1 s_1 t_2 s_2 \cdots s_N t_{N+1} $ となる文字列 $ t_i (1 \leq i \leq N + 1) $ が存在して,文字列 $ t_i (1 \leq i \leq N)$ は文字 $ s_i $ を含まない」となる $ T $ は何通りあるか
\end{block}
$ t_{N + 1} $ の文字数を $ j $ とすれば,重複組合せの考え方を用いて

\begin{center}
	$26^{j} \ 25^{K - j} \ \binom{K - j + N - 1}{N - 1}$
\end{center}
となり,最終的な答えはこれを $ 1 \leq j \leq K $ で足し上げる\\
全体として $O(N + K)$ で解けた
\end{frame}

\begin{frame}
\begin{alertblock}{Remark}
	\begin{itemize}
		\item 最後の式の導出は母関数を使ってもできる$ (1-25x)^{-N}(1-26x)^{-1} $ の $ x^K $ の係数が求めたいもの
		\item 「dp[i][j] := i文字目まで決めて$S$のj文字目まで含まれる場合の数」から計算量を落としてもいい
		\item 操作の完成形の場合の数は,「完成形の条件を見つける」「操作と完成形を一対一に対応させる」と上手くいきやすい
		\item OEIS(オンライン整数列大辞典)にあるらしい
	\end{itemize}
\end{alertblock}

\href{https://atcoder.jp/contests/abc171/submissions/14675270}{提出}

\end{frame}

%
%
\end{document}