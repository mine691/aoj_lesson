\documentclass[dvipdfmx,12pt]{beamer}% dvipdfmxしたい
%
\usepackage{amsmath,amssymb}
\usepackage{bm}
\usepackage{graphicx}
\usepackage{ascmac}
% pLaTeX文書; 文字コードはいつも通り
\usepackage{bxdpx-beamer}% dvipdfmxなので必要
\usepackage{hyperref} %リンクを有効にする

\renewcommand{\kanjifamilydefault}{\gtdefault}% 既定をゴシック体に
\usefonttheme{professionalfonts}
% あとは欧文の場合と同じ
\usetheme{Copenhagen}

%
\title{ABC174-F 解説}
\author{@mine691}
\date{\today}
\begin{document}
\maketitle
%

\section{問題概要}
\begin{frame} % スライド1pごとに frame を作る
\begin{block}{問題文の概略}
% ココに問題文を書く
長さ $ N $ の数列があり,$ i $ 番目の数は $ c_i $ です.\\
以下のようなクエリが $ Q $ 個与えられます.\\
\begin{itemize}
	\item 第 $ l $ 項目から第 $ r $ 項目までの部分列にある種類数を計算する.
\end{itemize}
\end{block}

\begin{exampleblock}{制約}
% ココに制約を箇条書きで書く
	\begin{itemize}
		\item 入力はすべて整数
		\item $1 \leq N, Q \leq 5 \times 10 ^ {5}$
		\item \alert{$1 \leq c_i \leq N$}
		\item $1 \leq l_i \leq r_i \leq N $
	\end{itemize}
\end{exampleblock}
\href{https://atcoder.jp/contests/abc174/tasks/abc174_f}{問題URL}
\end{frame}

\section{観察}

\begin{frame}
\begin{exampleblock}{具体例}
	例えば,\\
	$ N = 4, Q = 1 $ \\
	$ c = (1,2,1,3), q_1 = (1,3) $ \\
	なら ,部分列は $ (1,2,1) $ なので種類数は 2 \\
\end{exampleblock}
$ N,Q $ が大きいので,データ構造やクエリ処理を工夫したい.

\end{frame}

\section{考察}

\begin{frame}
\begin{block}{典型考察}
区間のオフラインクエリは,右端or左端でソートしてから処理するといい.
\end{block}
とりあえず,左端でソートをして考えてみる.\\
ところで,数列の種類数はどのように決まっているのだろうか.
\end{frame}

\begin{frame}
左からみて,初めて出てくる位置に印を付けてみる.\\
\begin{center}
\includegraphics[width=5cm]{abc174-f.png} \\
本質の図
\end{center}
左端を一つ進めるごとに初めて出てくる位置が更新されていて,この印の個数の総和が種類数になるので,良い感じのデータ構造があればOK
\end{frame}

\section{解法}
\begin{frame}
\begin{block}{実装について}
BIT でさっきの処理を行う.\\
初めて出てくる位置の更新は,数ごとのpriority queue / queue があればできる.
\end{block}

\href{https://atcoder.jp/contests/abc174/submissions/15722462}{提出コード}\\
この方針だと丁寧に書いて40行くらい(BITを除く)
\end{frame}

\begin{frame}
\begin{exampleblock}{類題}
\begin{itemize}
	\item $ N = 26 $ で種類数クエリと更新クエリ(\href{https://codeforces.com/contest/1234/problem/D}{CR590-D})
	\item $ K $ 個の重複を許す種類数クエリ(オンライン)(\href{https://codeforces.com/problemset/problem/813/E}{ECR22-E})
\end{itemize}
\end{exampleblock}

\begin{alertblock}{その他}
\begin{itemize}
	\item 問題名  Range Set Query を言い換えて [区間 種類数 クエリ] と検索すると,良い感じの記事が出てくる
	\item 制約が微妙にきつくて,定数倍で落ちたりする
	\item 同じ数が端の区間を列挙して,(クエリ区間の長さ)$-$(そのクエリ区間を含むような区間の個数)が答え.これはBITを使った平面走査でできる.
	\item 他にも Mo's algorithm でもできるが,TLが厳しめ
\end{itemize}
\end{alertblock}

\end{frame}

%
%
\end{document}
