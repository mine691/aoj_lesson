\documentclass[dvipdfmx,12pt]{beamer}% dvipdfmxしたい
%
\usepackage{amsmath,amssymb}
\usepackage{bm}
\usepackage{graphicx}
\usepackage{ascmac}
% pLaTeX文書; 文字コードはいつも通り
\usepackage{bxdpx-beamer}% dvipdfmxなので必要
\usepackage{hyperref} %リンクを有効にする
\newcommand{\thickhrulefill}{\leavevmode\leaders\hrule depth-1.2pt height 3.2pt\hfill\kern0pt}
\newcommand{\indicatewidth}[1]{\thickhrulefill{#1}\thickhrulefill}

\renewcommand{\kanjifamilydefault}{\gtdefault}% 既定をゴシック体に
\usefonttheme{professionalfonts}
% あとは欧文の場合と同じ
\usetheme{Copenhagen}

%
\title{ABC183-F Confluence 解説}
\author{mine691}
\date{\today}
\begin{document}
\maketitle
%

\section{問題}
\begin{frame} % スライド1pごとに frame を作る
\begin{block}{問題文}
% ココに問題文を書く
$ N $ 人の生徒がいる。生徒 $ i $ はクラス $ C_i $ に属している。
以下の $ Q $ 個のクエリが与えられる。
\begin{itemize}
	\item "1 a b" : 生徒 $ a $ を含むグループと、生徒 $ b $ を含むグループをマージする。
	\item "2 x y" : 生徒 $ x $ を含むグループのうち、クラス $ y $ に属している生徒の数を求める。
\end{itemize}
\end{block}

\begin{exampleblock}{制約}
% ココに制約を箇条書きで書く

	\begin{itemize}
		\item 入力はすべて整数
		\item $ 1 \leq N \leq 2 \times 10 ^ {5} $
		\item $ 1 \leq Q \leq 2 \times 10 ^ {5} $
		\item $ 1 \leq C_i, a, b, x, y \leq N $
	\end{itemize}

\end{exampleblock}
\end{frame}

\begin{frame}
クエリ 1 は dsu でできそうな見た目をしている。\\
クラスの情報を std::map で保持することにする。\\
マージの方法は、マージテク(weighted quick find)を使う。
\begin{alertblock}{集合をマージする一般的なテク}
	二つの集合をマージするとき、サイズの小さいほうを大きいほうへマージするようにすると、毎回の操作のならし計算量が $ O (\log N) $ になる。\\
	これは dsu の Union by size でも使われるテクニックである。 
\end{alertblock}

\end{frame}

\begin{frame}
マージテクを使うと、
\begin{itemize}
	\item クエリ 1 は マージ回数 $ O (\log N) $、std::map / dsu の操作に $ O (\log N) $ 
	\item クエリ 2 は std::map のアクセスに $ O (\log N) $ 
\end{itemize}
まとめると、$ O(N(\log N)^2+Q\log N) $ となる。
\end{frame}

\begin{frame}{別解1}
UnionFind で経路圧縮 + マージテクを使うと上手くマージできない?\\
そんなことはなくて、集合の代表元が親であることを知っていれば、マージ前後に変更を確認すると、 ACL の dsu でもできる。
\href{https://atcoder.jp/contests/abc183/submissions/18254918}{実装例}

\end{frame}

\begin{frame}{別解2}
平方分割をする。\\
生徒 $ a $ が属している集合のサイズが $ \sqrt{N} $ であるかを管理する。\\
1 クエリはサイズごとで場合分けをする。\\
2 クエリが $ O(\sqrt{N}) $ でできる。\\
なお実装は……。
\end{frame}

\end{document}