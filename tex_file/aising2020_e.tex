\documentclass[dvipdfmx,12pt]{beamer}% dvipdfmxしたい
%
\usepackage{amsmath,amssymb}
\usepackage{bm}
\usepackage{graphicx}
\usepackage{ascmac}
% pLaTeX文書; 文字コードはいつも通り
\usepackage{bxdpx-beamer}% dvipdfmxなので必要
\usepackage{hyperref} %リンクを有効にする

\renewcommand{\kanjifamilydefault}{\gtdefault}% 既定をゴシック体に
\usefonttheme{professionalfonts}
% あとは欧文の場合と同じ
\usetheme{Copenhagen}

%
\title{aising2020-E 解説}
\author{@mine691}
\date{\today}
\begin{document}
\maketitle
%

\section{問題概要}
\begin{frame} % スライド1pごとに frame を作る
\begin{block}{問題文}
% ココに問題文を書く
$ 1, 2, \cdots , N$ の番号がついたラクダ $ N $ 頭を一列に並べます.
ラクダ $ i $ は先頭から $ K_i $ 番目以内にいるときのうれしさは $ L_i $で,そうでない場合の嬉しさは $ R_i $です.
最適な配置により,ラクダのうれしさの総和の最大化してください.
\end{block}

\begin{exampleblock}{制約}
% ココに制約を箇条書きで書く

	\begin{itemize}
		\item 入力はすべて整数
		\item $1 \leq N \leq 2 \times 10 ^ {5}$
		\item $1 \leq K_i \leq N$
		\item $1 \leq L_i, R_i \leq 10 ^ {9}$
	\end{itemize}

\end{exampleblock}
\href{https://atcoder.jp/contests/aising2020/tasks/aising2020_e}{問題URL}
\end{frame}

\section{観察}

\begin{frame}
\begin{exampleblock}{具体例}
	例えば,\\
	$ N = 3 $ \\
	$ (K_1, L_1, R_1) = (2, 93, 78) $ \ $ (K_2, L_2, R_2) = (1, 71, 59) $ \\
	$ (K_3, L_3, R_3) = (3, 57, 96) $ \\
	なら, $2, 1, 3$ の順で並べるのがいい.\\
	うれしさの総和は, $ 93 + 71 + 57 = 221 $ が最大.
\end{exampleblock}
一見フロー(ラクダと位置の重み付き最大マッチング,最小費用流)に見えるけど制約的に無理.\\
なにか良い性質を利用したい.
\end{frame}

\begin{frame}

\begin{alertblock}{性質}
総和の下界はどれくらいだろうか?\\
$ K_i $ 以内かに依らず $ \min(L_i, R_i) $ のうれしさは得られる.\\
つまり,最初に $ \Sigma_i \min(L_i, R_i) $ を計算しておけば, $ D_i := | L_i - R_i | $ として,
$ (0, D_i) $ か $ (D_i, 0) $ の形にできる.
\end{alertblock}
	$ (K_1, L_1, R_1) = (2, 93, 78) $ \\
	$ (K_2, L_2, R_2) = (1, 71, 59) $ \\
	$ (K_3, L_3, R_3) = (3, 57, 96) $ \\
を\\
	$ (L_1, D_1) = (2, 15) $ \\
	$ (L_2, D_2) = (1, 12) $ \\
	$ (L_3, D_3) = (3, 39) $ \\
みたいに書ける.(ただし,$ L_i, R_i $ の大小は省略)

\end{frame}

\begin{frame}
\begin{alertblock}{性質}
最適解の列を観察してみる.
\begin{itemize}
	\item $ L_i > R_i $ なら左側にいる.
	\item $ L_i < R_i $ なら右側にいる.
\end{itemize}
それはそう.そうでなければ改善できるから.\\
左・右側にいるほうがコストが高いなら,左・右側にいるべき.
\end{alertblock}
	$ (K_1, L_1, R_1) = (2, 93, 78) $ \\
	$ (K_2, L_2, R_2) = (1, 71, 59) $ \\
	$ (K_3, L_3, R_3) = (3, 57, 96) $ \\
で $ 3, 2, 1$ だったとき($57 + 59 + 78$),3 と 1 を swap して総和を増やせる.\\
これで,左右独立にして考えられる.
\end{frame}

\begin{frame}
以下では,$ \forall i, L_i > R_i $ を仮定する.(左右それぞれで同じことをすればいい)\\
このとき,以下のような貪欲法ができる.(証明は難しくない)
\begin{block}{貪欲法}
	ラクダの位置を後ろから決めていく.($ j = N, \cdots 2, 1$)\\
	$ j \leq K_i $ のもののうち, $ D_i $ が大きいものから貪欲に選んでいく.(priority\_queueでやる)
\end{block}
この方法だと,$ O(N \log N)$ になって間に合う.(コードを見たほうが分かりやすい)\\
\href{https://atcoder.jp/contests/aising2020/submissions/15270723}{提出コード}
\end{frame}

\begin{frame}{余談}
\begin{itemize}
	\item ラストの部分の証明は「マトロイドなので」で終わらせられる.(「$ K_i $ 以内に入れることができるもの」で部分集合を取る)
	\item マトロイドについては,\href{https://drken1215.hatenablog.com/entry/20121212/1355280288}{けんちょんさんの記事}がおすすめ
	\item 位置を前から決めてもOK(直感的に分かりにくいかも?)(溢れた数を捨てていく)
\end{itemize}

\end{frame}

%
%
\end{document}
